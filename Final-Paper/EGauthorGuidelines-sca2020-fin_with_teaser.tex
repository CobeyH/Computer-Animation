% ---------------------------------------------------------------------------
% Author guideline and sample document for EG publication using LaTeX2e input
% D.Fellner, v1.15, Dec 14, 2018

\documentclass{egpubl}
\usepackage{sca2020}
 
% --- for  Annual CONFERENCE
% \ConferenceSubmission   % uncomment for Conference submission
% \ConferencePaper        % uncomment for (final) Conference Paper
% \STAR                   % uncomment for STAR contribution
% \Tutorial               % uncomment for Tutorial contribution
% \ShortPresentation      % uncomment for (final) Short Conference Presentation
% \Areas                  % uncomment for Areas contribution
% \MedicalPrize           % uncomment for Medical Prize contribution
% \Education              % uncomment for Education contribution
% \Poster                 % uncomment for Poster contribution
% \DC                     % uncomment for Doctoral Consortium
%
% --- for  CGF Journal
% \JournalSubmission    % uncomment for submission to Computer Graphics Forum
% \JournalPaper         % uncomment for final version of Journal Paper
%
% --- for  CGF Journal: special issue
% \SpecialIssueSubmission    % uncomment for submission to , special issue
\SpecialIssuePaper         % uncomment for final version of Computer Graphics Forum, special issue
%                          % EuroVis, SGP, Rendering, PG
% --- for  EG Workshop Proceedings
% \WsSubmission      % uncomment for submission to EG Workshop
% \WsPaper           % uncomment for final version of EG Workshop contribution
% \WsSubmissionJoint % for joint events, for example ICAT-EGVE
% \WsPaperJoint      % for joint events, for example ICAT-EGVE
% \Expressive        % for SBIM, CAe, NPAR
% \DigitalHeritagePaper
% \PaperL2P          % for events EG only asks for License to Publish

% --- for EuroVis 
% for full papers use \SpecialIssuePaper
% \STAREurovis   % for EuroVis additional material 
% \EuroVisPoster % for EuroVis additional material 
% \EuroVisShort  % for EuroVis additional material

% !! *please* don't change anything above
% !! unless you REALLY know what you are doing
% ------------------------------------------------------------------------
\usepackage[T1]{fontenc}
\usepackage{dfadobe}  

%\usepackage{cite}  % comment out for biblatex with backend=biber 
% ---------------------------
\biberVersion
\BibtexOrBiblatex
\usepackage[backend=biber,bibstyle=EG,citestyle=alphabetic,backref=true]{biblatex} 
\addbibresource{egbibsample.bib}
% ---------------------------  
\electronicVersion
\PrintedOrElectronic

% for including postscript figures
% mind: package option 'draft' will replace PS figure by a filename within a frame
\ifpdf \usepackage[pdftex]{graphicx} \pdfcompresslevel=9
\else \usepackage[dvips]{graphicx} \fi

\usepackage{egweblnk} 
% end of prologue

\begin{document}
\section{Abstract}
In this project, a simple simulator was created to model multi-agent evolution. Evolution is typically very slow and difficult to observe on the timescale of a human life. However, computer animation provides an excellent framework for modelling evolution with very rapid generations and a high mutation rate to observe the effects quickly. A flocking method called boids was utilized to create interesting looking flocks that could work together to overcome a predator. Boids provides a very simple set of rules that produce realistic looking behaviour with basic rules.
\par
The boids were extented by creating multiple distinct flocks which each had members created with a random value for mass, speed, cohesion, alignment and separation. Each of these attributes would influence the survival rate of the boids and the interactions with other boids in their flock and the predators. A food system was also added force the boids to eat in order to survive. Each simulation tick increases the boids hunger and their desire to find food.
\par
If a boid dies, either from starvation or from a predator boid, then a new boid is created by randomly selecting attributes from two random boids. These attributes also has a chance to increase or decrease by a small amount through a mutation mechanism.
\par
This method produced the desired results in which the average value of each attribute would increase or decrease over time until the optimal value was found for each. \textbf{FINISH RESULTS} 
\section{Introduction}
\section{Related Work}
The majority of this project was created from scratch with inspiration drawn from real life. However, several sources were used to create the inital boid behaviour and perform performance optimizations to increase the number of boids the application could support.
\par
Craig W. Reynolds' paper entitles "Flocks, Herds, and School: A Distributed Behavioral Model" provided the groundwork for how boid flocking forces work and the inspiration for the project \textbf{CITE HERE}. In addition, a spacial subdivision optimization data structure called Quadtrees were implemented in this project to improve the performance. This data structure was implemented with the help of a Wikipedia article that provided pseudocode to get started with the basic functions \textbf{CITE HERE}.
\section{Overview}
In this project, an evolution simulator was created using Boids to model flocking behaviour for the agents. The project will be described as several different subsystems that each work together in the completed product. 
\subsection{Boids}

\subsection{Evolution}
\subsection{QuadTrees}
\section{Evaluation}
\section{Conclusion}
\printbibliography                
\end{document}
